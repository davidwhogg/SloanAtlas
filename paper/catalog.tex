\documentclass[12pt,preprint,pdftex]{aastex}
\newcounter{address}
\newcommand{\project}[1]{\textsl{#1}}
\newcommand{\units}[1]{\mathrm{#1}}
\renewcommand{\mag}{\units{mag}}
\renewcommand{\arcmin}{\units{arcmin}}
\renewcommand{\arcsec}{\units{arcsec}}
\newcommand{\rfifty}{r_{50}}
\newcommand{\rninety}{r_{90}}
\newcommand{\conc}{C}
\newcommand{\foreign}[1]{\emph{#1}}
\newcommand{\etal}{\foreign{et\,al.}}
\begin{document}

\title{
       The SDSS Large Galaxy Atlas
      }
\author{
        Ekta Patel\altaffilmark{\ref{CCPP}},
        David Mykytyn\altaffilmark{\ref{CCPP}},
        David W. Hogg\altaffilmark{\ref{CCPP},\ref{MPIA},\ref{email}},
        Dustin Lang\altaffilmark{\ref{CMU}}
       }
\setcounter{address}{1}
\altaffiltext{\theaddress}{\stepcounter{address}\label{CCPP} Center
  for Cosmology and Particle Physics, Department of Physics, New York
  University, 4 Washington Place, New York, NY 10003}
\altaffiltext{\theaddress}{\stepcounter{address}\label{MPIA}
  Max-Planck-Institut f\"ur Astronomie, K\"onigstuhl 17, D-69117
  Heidelberg, Germany}
\altaffiltext{\theaddress}{\stepcounter{address}\label{email} To whom
  correspondence should be addressed: \texttt{david.hogg@nyu.edu}}
\altaffiltext{\theaddress}{\stepcounter{address}\label{CMU} Carnegie
  Mellon University}

\begin{abstract}
We present the \project{Sloan Digital Sky Survey Large Galaxy Atlas}, which contains
accurate positions and photometry for galaxies with half-light diameter
greater than $1~\arcmin$. Galaxies of this size have rarely been measured with great accuracy. We present a table of measurements for such galaxies including their positions, colors, magnitudes, sizes, concentrations, surface brightness and axis ratios. Our input data has been taken largely from SDSS as well as the Third Reference Catalog of Bright Galaxies(RC3), the Two Micron All Sky Survey Large Galaxy Atlas and the NASA-Sloan Atlas. A description of the methods used to make measurements of the galaxies provided in this table can be found in the companion paper for this project, Mykytyn et al. 2013. We specifically highlight those large galaxies which have not been found in RC3, but, have been included in the NASA-Sloan Atlas with $50\%$ light radii $> 30\,\arcsec$. Some examples of these include galaxies from the Messier catalog which have been measured as large as $\sim 260\,\arcsec$ while retaining accurate galactic properties. We have stopped at $\sim 260\,\arcsec$ for matters of necessity with regards to future projects.

\end{abstract}

\section{Introduction}

The \project{SDSS} never made good photometry for the brightest galaxies because it oversubtracts the sky for very large objects, but it did make beautiful images. The Nasa-Sloan Atlas(NSA) rephotometered all redshifts for galaxies within the SDSS Data Release 8. It takes the images from the \textit{ugriz} bands of SDSS and the NUV and FUV bands from \textit{GALEX}.\footnote{\url{www.nsatlas.org}} The methods used in the NSA is an improvement to those used by SDSS. It begins with and estimation of the sky from an processed SDSS image and creates masks for all bright sources in and around the image. A spline is fit to the unmasked data with constraints that make the process run even for images where heavy masking is needed \citep{blanton11}. 
% how does NSA improve the SDSS background? What is the SDSS method? How do we correct for sky using tractor?

The RC3 catalog, which contributes the most galaxies to our data set, gives methods of measurements in Volume I of the catalog. Positions for the galaxies contained in RC3 were measured using the SAO Star Catalog and the FK4 coordinate system, but are listed by the 2000.0 equinox positions. A large number of positions were also taken from the CGCG, UGC, and MCG catalogs. $D_{25}$ was measured at the 25.0 B-m/ss level in units of 0.1 arcminutes. The isophotal diameters are largely taken from three sources including derivations from de Vaucouleur's photoelectrically calibrated photographic and CCD photometry, interpolation of R.Buta's photoelectric growth curves, and from the photoelectric calibration of surface photometry by Lauberts and Valentijn from which the largest contributions are made to RC3's $D_{25}$ values. Galactic extinction was measured using the Burstein \& Heiles method, which involves a combination of local galacatic HI column densities and faint galaxy counts. Internal extinction is measured by \begin{equation} A_i= \alpha(T)log(R_{25}) \end{equation} where $\alpha(T)$ is a coefficient dependent on morphological type and $R_{25}$ is the axis ratio. Various magnitudes were measured and included in the RC3 catalog. The total photoelectric magnitudes, $B_T$, were found using two methods which are the extrapolation of photoelectric aperture photometry using standard curves and the extrapolation of the calibrated surface photometry. Two different surface brightnesses are included. One measures the mean effective surface brightness with the effective aperture, $A_e$. The other method measures the mean surface brightness within the $D_{25}$ ellipse, using the the axis ratio,$R_{25}$\citep{rc3}.

The 2MASS Large Galaxy Atlas(LGA) provides near-infrared photometry for large galaxies to connect the gap between optical wavelengths and longer wavelengths. We have checked for galaxies in the LGA that were not included in RC3 down to a radius of $1\arcmin$, which was the lower size limit for galaxies included in this catalog. The radius measurement in 2MASS was based on the K-band isophotal properties \citep{jarrett03}.  

The \textit{GALEX} Ultraviolet Atlas of Nearby Galaxies provides photometry, surface brightness and color profiles for 1034 nearby galaxies with a diameter larger than $1~\arcmin$ and a surface brightness in the b-band, $u_B$=25 mag arcsec$^{-2}$ . The galaxies chosen for this data set come from the \textit{GALEX} Nearby Galaxies Survey and other equivalent\textit{GALEX} initiatives that viewed the same fields of the sky in the far-ultra violet(FUV) and near-ultra violet(NUV) bands. All galaxies with a diameter larger than $1~\arcmin$ from RC3 were also added to the sample. A total of 81 out of 1136 galaxies were not included in the Atlas because they were in regions of very high background, high Galactic extinction, extremely low UV surface brightness, or simply had images that were not of sufficient quality for analysis. The surface brightness was measured using the central position, ellipticity, and position angle of the D25 ellipse. It is computed to within the annuli of ellipse's central angle increasing out by $6~\arcsec$ from the major-axis until it reaches at least 1.5 times D25. Point sources in the images with colors redder than FUV-NUV=1 were automatically masked and visually checked. After the SExtractor-detected sources were masked within a 5x5 pixel range of each source, the mean of the sky was used as the estimate of the background due to low background in FUV. Galactic color excess is taken from \cite{schlegel98} and used with the extinction law from \cite{cardelli}. The surface brightness profiles were then used to compute asymptotic magnitudes, colors, luminosities and concentrations. Magnitudes, effective radii, and colors were found by extrapolating growth curves to infinity and performing error-weighted linear fits. The morphology of the UV surface brightness profiles were determined visually using a two-letter naming scheme, the first describing the outer profile and the second describing the inner profile \citep{gdp06}.

%HOGG: discuss general issue that our photometry is being done the way it is to meet particular goals

\section{Data and input catalogs}

%Hogg: Stuff about the \project{SDSS} imaging ...

The initial input catalog was formed out of the union of the Third Reference Catalog of Bright Galaxies (RC3), the NASA-Sloan Atlas (NSA), and the 2MASS Large Galaxy Atlas (LGA). Objects were searched for in those catalogs with an angular diameter greater than or equal to two arcminutes. For the Third Reference Catalog the cut made was based on the measured apparent major isophotal diameter of the surface brightness $\mu_{B}$ greater than 1 arcminute. For the NASA-Sloan Atlas, we checked for any objects of significant size that were
also not in RC3. The cut was for objects with a 50\% light radius of
the SERSIC fit greater than 30 arcseconds. The objects were then
examined by eye to determine which ones were actual large galaxies,
and which were errors in the NSA. Example errors included stars
causing saturation, interstellar medium, and nebulae. Some interesting galaxies that were not contained in RC3 were found and measured and have been included in the Atlas. Some of these images contained two galaxies, which we fit individually. We also checked
the 2MASS Large Galaxy Atlas but no galaxies of the correct size not
already in RC3 were found. The only objects found were either not in
the Sloan Survey, or were globular clusters. The cut for 2MASS was
based on the isophotal fiduciary radius in the K-band, which was
checked to a radius of 1 arcminute.

\section{Method}

%Refer to the method paper but summarize highlights ...

For the largest of galaxies, SDSS provided many fields where each field contained images in each of the \textit{ugriz}. Examples of these include the Messier galaxies. For those galaxies which had between 10 and 20 images, the pixels in the images were scaled down by a factor of 2. Galaxies with 20 to 40 images each were scaled down by a factor of 4. Galaxies with 40 to 80 images were scaled down by a factor of 8 and those with more than 80 images could not be run through the \textit{Tractor} successfully. The binning of the pixels in these images did, however, provide sufficient accuracy to within a few percent for the colors in comparison to colors resulting from no binning. Other cases of galaxies that had to be treated specially include those in which there was a pair of galaxies in one image. These scenarios were manually measured as two individual galaxies. Some examples include M51 and the Mice Galaxies in which cases the secondary measurements seemed to be within a few percent error of the dense region in the color-color plot in Figure 2(are we plotting these on the color-color to show it?). Another special case is that of a bright star whose masking caused a part or whole of the galaxy to be masked with it. In some cases, these were re-run manually, while those in which the whole galaxy was masked were discarded.  Additionally, we observed those cases in which galaxies were off-center in the images and re-ran those manually with an initialization beginning at the center of the galaxy rather than the center of the image. 

With the raw results from the $Tractor$, we first applied extinction corrections as given by Schlegel et al. 1998 using the \textbf{astropysics} package.\footnote{\url{packages.python.org/Astropysics}} The get\_sfd\_dust function in the obstools module of the {\bf astropysics} package allows us to retrieve the reddening value E(B-V) from the dust maps by converting the RA and Dec that we measure to galactic coordinates. The reddening values are then multiplied by the appropriate coefficients A(V)/E(B-V) per Sloan bandpass to obtain the total extinction to be subtracted from the individual magnitudes of each bandpass. The coefficients are the following: $u=5.155,\;g=3.793,\; r=2.751,\; i=2.086,\; z=1.479$. With the corrected magnitudes, we were then able to compute surface brightness with \begin{equation} \mu_{50,i}=m_i+2.5log(\pi{r_{50}}^2) \end{equation} where $m_i$ is the corrected magnitude in the i-band and $r_{50,i}$ is the half-light radius in the i-band. For a description of how the radii, concentrations, magnitudes, and axis ratios were computed see the methodology in Mykytyn et al. 2013. 


%A few examples showing what works and what doesn't ...

\section{Catalog}
The table of data contains the RA, Dec, and name given by each galaxy's source catalog(RC3 or Nasa-Sloan Atlas).For the measured positions of all galaxies included in the Atlas, see Figure 1. The data from the composite galaxies includes the RA, Dec, magnitudes, half-light radii, colors, etc. See Table 1. Those galaxies which we have denoted with flags... 

%Also exceptions tables ...

Galaxies that have been marked with exceptions include those which have been discussed in section 3. These are galaxies which have been determined to give poor results by eye and were then re-run manually. In some cases, the second result provided valuable measurements, while in others the data was not included for continuous poor measurements. See our comments from visual inspections in $large\_galaxies.txt$ and in $large\_NSA\_galaxies.txt$

\section{Discussion}

The Owl Nebula was included in the list of objects that are not in RC3, but have been included in NASA-Sloan Atlas. Though we are not particularly interested in nebulae, the fit for the Owl Nebula is succesfully. Notice also that our measurements in color are consistent throughout our sample from very small to very large galaxies. See Figure 5. 

%HOGG: discuss general issue that our photometry is being done the way it is to meet certain particular goals

\acknowledgements
It is a pleasure to thank Michael Blanton(NYU), Dan Foreman-Mackey(NYU), Douglas Finkbeiner(Harvard), Thomas Robitaille(MPIA,asked him about SFD interface), Benjamin Weaver(NYU), and Stephane Courteau(Queen's College) for their time and conversations with regards to the Atlas. Funding for the SDSS Large Galaxy Atlas has been provided by New York University's College of Arts and Science Dean's Undergraduate Research Fund.This work was supported in part by The Princeton University Press,
NASA (grant NNX12AI50G) and the NSF (grant IIS-1124794).

%\acknowledgements It is a pleasure to thank
%  Jim Gunn (Princeton),
%  Robert Lupton (Princeton), and
%  David Spergel (Princeton)
%for valuable discussions.

%...HOGG: Insert relevant SDSS boilerplate here...

The code for the methods used to make the measurements in this catalog can be found at: www.github/davidwhogg/SloanAtlas

\begin{thebibliography}{70}
\bibitem[Blanton \etal(2011)]{blanton11}
Blanton,~M., 2011, \apj~142, 31
%{Blanton, M.R. et al., Improved background subtraction for the Sloan Digital Sky Survey images, 2011, 

\bibitem[Cardelli \etal(1989)]{cardelli}
Cardelli,~J.~A., Clayton,~G.~C.,\& Mathis,~J.~S., 1989 \apj~345,245

\bibitem[Gil de Paz \etal(2006)]{gdp06}
% The $GALEX$ Ultraviolet Atlas of Nearby Galaxies, 
Gil de Paz,~A., 2006 \aj~173, 185

\bibitem[Jarrett \etal(2003)]{jarrett03}
% The Hubble Tuning Fork Strikes a New Note: The 2MASS Large Galaxy Atlas, 2003, 
Jarrett,~T.~H., 2003, \aj~125, 525

\bibitem[Schlegel \etal(1998)]{schlegel98}
Schlegel,~D.~J., Finkbeiner,~D.~P., Davis,~M., 1998, \apj~500, 525
% Maps of Dust IR Emission for Use in Estimation of Reddening and CMBR Foreground

\bibitem[de Vaucouleurs (1991)]{rc3}
de Vaucouleurs, ~G., de Vaucouleurs,~A., Corwin, ~H.~G., Buta, ~R.~J., Paturel, ~G., Fouqu\'{e}, ~P., 1991, Volume I
\end{thebibliography}
\clearpage

\begin{figure}
\centering
\includegraphics[trim= 15mm 0mm 0mm 0mm]{ra_dec.pdf}
\caption{RA vs. DEC for all galaxies in our data set. Note that the galaxies measured are mostly contained in the northern galactic hemisphere as a result of the location of the telescope used for the Sloan Digital Sky Survey Data at Apache Point Observatory, New Mexico.}
\end{figure}

\begin{figure}
\centering
\includegraphics[trim= 15mm 0mm 0mm 0mm]{gi_ri.pdf}
\caption{The g-i vs. r-i colors for all galaxies.}
\end{figure}


\begin{figure}
\centering
\includegraphics[trim= 15mm 0mm 0mm 0mm]{r50_i.pdf}
\caption{The half-light radius of the i-band flux vs. the i-band magnitude. The i-band magnitudes given are corrected using the Schlegel and Finkbeiner dust maps (cite).}
\end{figure}

\begin{figure}
\centering
\includegraphics[trim= 15mm 0mm 0mm 0mm]{gi_mu_c.pdf}
\caption{A comparison of the g-i colors vs. surface brightness in the i-band vs. the ratio of half-light radius to 90\% light radius. The vertical lines provide the quartiles of the data set.}
\end{figure}

\begin{figure}
\centering
\includegraphics[trim= 15mm 0mm 0mm 0mm]{consistency_sp.pdf}
\caption{Consistency check of colors throughout the range of the data sample by angular size.}
\end{figure}


\end{document}
