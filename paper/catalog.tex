\documentclass[12pt,preprint,dvipdf]{aastex}
\newcounter{address}
\newcommand{\project}[1]{\textsl{#1}}
\newcommand{\units}[1]{\mathrm{#1}}
\renewcommand{\mag}{\units{mag}}
\renewcommand{\arcmin}{\units{arcmin}}
\newcommand{\rfifty}{r_{50}}
\newcommand{\rninety}{r_{90}}
\newcommand{\conc}{C}

\begin{document}

\title{
       The SDSS Large Galaxy Atlas
      }
\author{
        Ekta Patel\altaffilmark{\ref{CCPP}},
        David Mykytyn\altaffilmark{\ref{CCPP}},
        David W. Hogg\altaffilmark{\ref{CCPP},\ref{MPIA},\ref{email}},
        Dustin Lang\altaffilmark{\ref{CMU}}
       }
\setcounter{address}{1}
\altaffiltext{\theaddress}{\stepcounter{address}\label{CCPP} Center
  for Cosmology and Particle Physics, Department of Physics, New York
  University, 4 Washington Place, New York, NY 10003}
\altaffiltext{\theaddress}{\stepcounter{address}\label{MPIA}
  Max-Planck-Institut f\"ur Astronomie, K\"onigstuhl 17, D-69117
  Heidelberg, Germany}
\altaffiltext{\theaddress}{\stepcounter{address}\label{email} To whom
  correspondence should be addressed: \texttt{david.hogg@nyu.edu}}
\altaffiltext{\theaddress}{\stepcounter{address}\label{CMU} Carnegie
  Mellon University}

\begin{abstract}
We present the \project{SDSS Large Galaxy Atlas}, which contains
accurate positions and photometry for galaxies with half-light radii
greater than $1~\arcmin$.
\end{abstract}

\section{Introduction}

... The \project{SDSS} never made good photometry for the brightest galaxies ... The NS Atlas ...

... The \project{SDSS} made beautiful images ...

... Related work:  2MASS ... anything else? ...

\section{Data and input catalogs}

... Hogg: Stuff about the \project{SDSS} imaging ...

... Mykytyn: How did you make the input catalog(s)? ...:  The input catalog was formed out of the union of the Third Reference Catalog of Bright Galaxies (RC3), the NASA-Sloan Atlas (NSA), and the 2MASS Large Galaxy Atlas (LGA). Objects were searched for in those catalogs with an angular diameter greater than or equal to two arcminutes. For the Third Reference Catalog the cut made was based on the measured apparent major isophotal diameter of the surface brightness $\mu_{B}$ greater than 2 arcminutes. For the NASA-Sloan Atlas the cut made was based on the 50\% light radius of the SERSIC fit. Finally for the 2MASS Large Galaxy Atlas the cut was made based on the isophotal fiduciary radius in the K-band, again cutting to a radius of 1 arcminute.

\section{Method}

... Refer to the method paper but summarize highlights ...

... Output vetting and re-running and manual tweaking---what did we do and why? ...

... A few examples showing what works and what doesn't ...

\section{Catalog}

... The big table ...

... Also exceptions tables ...

\section{Discussion}

\end{document}
