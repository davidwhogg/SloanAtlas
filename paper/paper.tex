\documentclass{article}
\newcommand{\units}[1]{\mathrm{#1}}
\renewcommand{\mag}{\units{mag}}

\newcommand{\rfifty}{r_{50}}
\newcommand{\rninety}{r_{90}}
\newcommand{\conc}{C}

\begin{document}
For the purpose of measuring and optimizing on a single galaxy at a
time, all other objects were removed from within a circle centered on
the galaxy center and radius, both given by the RC3 catalog (which
radius?) for that particular catalog. This was done rather than
working within the SDSS catalog because SDSS frequently split up large
galaxies into numerous small ones and so it was much easier to simply
start over with a single galaxy. As initialization, that galaxy was
given magnitudes of 15 in every band and a radius given by the RC3
catalog.

In order to deal with the problems of stars that overlap with the
galaxy which we are interested in, we decided to mask them out. The
stars were found using the data in the TYCHO2 star catalog using the
spherematch function of astrometry.net. The stars were masked out to a
radius of
\begin{equation}
\theta_{\mathrm{exclude}} = \max(25, 25\,2^{[11\,\mag-V]}
\end{equation}
where $V$ is the visual magnitude given by TYCHO2
I'm not sure why these values, I got them from Dustin)

The iteratively-reweighted least squares process is used to update the
inverse variances after each iteration of optimization. The formula
used is: \begin{equation} \frac{1}{\sigma}=\frac{Q^2}{Q^2+\chi^2}
\end{equation}
where $Q$ is the number of ``sigma'' at which the residual saturates.

We start the fitting with an exponential galaxy and then optimize
simply that galaxy six times. Following that, we switch to a composite
galaxy, with both exponential and deVaucouleur components. These
components are free to vary separately in order to best fit the
data. This new composite galaxy is also optimized six times by itself.

In order to measure the half-light radius $\rfifty$, the 90 percent
radius $\rninety$ and the concentration parameter $\conc\equiv
\rninety/\rfifty$, we took the model of the galaxy that had been
optimized and places it in an empty frame within the tractor. Next,
over a series of increasing intervals of radii we summed up the total
flux within the circles and then interpolated to find $\rfifty$ and $\rninety$.

\end{document}

